\documentclass[11pt, oneside]{article}   	% use "amsart" instead of "article" for AMSLaTeX format
\usepackage[margin=3.4cm]{geometry}               		% See geometry.pdf to learn the layout options. There are lots.
\geometry{letterpaper}                   		% ... or a4paper or a5paper or ... 
\usepackage{graphicx}				% Use pdf, png, jpg, or eps§ with pdflatex; use eps in DVI mode
\usepackage{hyperref}
\usepackage{amssymb}
\usepackage{caption}
\usepackage{float}
\usepackage{setspace}
\usepackage{esvect}
\usepackage{url}
\usepackage{bm}
\usepackage{amsmath}
\usepackage{pythonhighlight}

%\setcounter{secnumdepth}{0}
\onehalfspacing

\begin{document}
\begin{titlepage}

	\center
	\textsc{\LARGE University of Calgary}\\[1.5cm]
	\vfill
	{\huge\bfseries Final Project}\\[0.4cm]
	{\huge\bfseries ENGO 623: Inertial Navigation}
	
	\vfill\vfill\vfill\vfill
	
	Adam \textsc{Smith} (30031453)\\
	
	\vfill\vfill\vfill
	{\large \today}
	
	\vfill\vfill\vfill
	\includegraphics[width=8cm,]{img/schulich.png}\\[1cm]	
\end{titlepage}

\section{Introduction}
In this project, we implement a Python module for mechanization of IMU data in the local level frame.  The module performs alignment for a user-specified time and then begins computing navigation information. Errors in attitude, position, and velocity are tracked throughout the mechanization and reported at the end.

\section{Mechanization Module Implementation}

\subsection{The INSMechanization class}
Mechanization is implemented in the \pyth{INSMechanization} class in the file \pyth{mechanization.py}.  This module is written using the scientific computing package \href{https://numpy.org}{\textit{numpy}} to allow for easy array operations.  To simplify computation, all calculations are performed in SI base or derived units. As such, the module requires all input information to be in SI base or derived units.  

To initialize the mechanization module, create an instance of the \pyth{INSMechanization} class, passing in the following arguments:
\begin{itemize}
	\item \pyth{h0} | Float \newline Initial height in meters above the ellipsoid. Required.
	\item \pyth{lat0} | Float \newline Initial latitude in radians. Required.
	\item \pyth{long0} | Float \newline Initial longitude in radians. Required.
	\item \pyth{accel_bias} | Float or Callable \newline Accelerometer bias in m/s$^2$. If callable, it should compute the accelerometer bias based on the local gravity. Default: 0
	\item \pyth{gyro_bias} | Float \newline Gyroscope bias in rad/s. Default: 0
	\item \pyth{accel_sf} | Float or Numpy Array \newline Accelerometer scale factor. Default: 0
	\item \pyth{gyro_sf} | Float or Numpy Array \newline Gyroscope scale factor. Default: 0
	\item \pyth{accel_no} | Numpy Array \newline Accelerometer non-orthogonality matrix in radians. Default: $3\times3$ zero matrix
	\item \pyth{gyro_no} | Numpy Array \newline Gyroscope non-orthogonality matrix in radians. Default: $3\times3$ zero matrix
	\item \pyth{vrw} | Float \newline Velocity random walk of the accelerometer in m/s$^\frac32$. Default: 0
	\item \pyth{arw} | Float \newline Angle random walk of the gyroscope in rad/s$^\frac12$. Default: 0
	\item \pyth{accel_corr_time} | Float \newline Accelerometer correlation time in seconds. Default: 0
	\item \pyth{gyro_corr_time} | Float \newline Gyroscope correlation time in seconds. Default: 0
	\item \pyth{accel_bias_instability} | Float or Callable \newline Accelerometer bias instability in m/s$^2$. If callable, it should compute the accelerometer bias based on the local gravity. Default: 0
	\item \pyth{gyro_bias_instability} | Float \newline Gyroscope bias instability in rad/s. Default: 0
	\item \pyth{alignment_time} | Float \newline Time duration in seconds to perform alignment before beginning navigation. Default: 0
\end{itemize}
The mechanization can then be run using the \pyth{process_measurement} method of the \pyth{INSMechanization} class.  This method accepts a single argument \pyth{measurement}, which represents a single measurement from the IMU and is a one-dimensional numpy array of length 7 consisting of the current time step in seconds, the x-, y-, and z-gyroscope measurements in rad/s, and the x-, y-, and z-accelerometer measurements in m/s$^2$.  

At any time, navigation parameters can be obtained through the \pyth{get_params} method.  This method returns the time, position, velocity, and attitude information as computed by the mechanization module at the most recent time step, as well as whether or not alignment has completed.  If the argument \pyth{degrees} is set to true, values in radians will be converted to degrees before being returned.

The following sections describe the implementation of the mechanization class in more detail.

\subsection{Alignment}
When a measurement is received in the \pyth{process_measurement} method, we check whether or not alignment has completed by checking the \pyth{alignment_complete} flag, which is initially set to False.  If alignment is not complete, the \pyth{align} method is called, passing in the current measurement.  The align method then uses the current timestamp (adjusted, if necessary, such that the first measurement received always has time 0) and the user-specified alignment time to determine whether or not alignment will be complete in this iteration.  If not, align then calls the \pyth{compensate_errors_and_compute_params} method (described in section 2.3) to compensate for deterministic errors in the accelerometer and gyroscope measurements.  The compensated values are then added to running totals, and a counter describing the number of times the \pyth{align} method has been called is incremented.  No further computation occurs.

Once \pyth{align} determines that alignment is complete, it sets the \pyth{alignment_complete} flag to True, then computes the mean of the measurements received during the alignment time by dividing the totals by the number of iterations.  The roll, pitch, and azimuth of the IMU are then calculated via
$$r=-\textrm{sign}(f_z)\sin^{-1}\frac{f_x}{g}$$
$$p=\textrm{sign}(f_z)\sin^{-1}\frac{f_y}{g}$$
$$A=\tan^{-1}\frac{-\omega_x}{\omega_y}$$
where $f$ and $\omega$ are the mean accelerometer and gyroscope measurements during alignment, respectively, and $g$ is the local gravity.  These values are used to compute the rotation of the IMU w.r.t. the LLF as
$$ R_b^l=\begin{pmatrix}
\cos A \cos r + \sin A \sin r \sin p & \sin A \cos p & \cos A \sin r - \sin A \cos r \sin p \\
\cos A \sin r \sin p - \sin A \cos r & \cos A \cos p & -\sin A \sin r - \cos A \cos r \sin p \\
-\cos p \sin r & \sin p & \cos p \cos r
\end{pmatrix}.$$
The quaternion describing this rotation is the calculated using the elements of $R_b^l$:
$$Q=\begin{pmatrix}
(r_{32}-r_{23})/4q_4 \\
(r_{13}-r_{32})/4q_4 \\
(r_{21}-r_{12})/4q_4 \\
q_4
\end{pmatrix}$$
where
$$q_4=\frac12\sqrt{1+r_{11}+r_{22}+r_{33}}$$
This quaternion is then divided component-wise by its Euclidean norm, which allows us to recompute the rotation matrix $R_b^l$ and ensure it is orthogonal.  $R_b^l$ can be recovered from the normalized quaternion via
$$ R_b^l=\begin{pmatrix}
q_1^2-q_2^2-q_3^2+q_4^2 & 2(q_1q_2-q_3q_4) & 2(q_1q_3+q_2q_4) \\
(q_1q_2+q_3q_4) & -q_1^2+q_2^2-q_3^2+q_4^2 & 2(q_2q_3-q_1q_4) \\
2(q_1q_3-q_2q_4) &  2(q_2q_3+q_1q_4) & -q_1^2-q_2^2+q_3^2+q_4^2
\end{pmatrix}.$$

Once alignment is complete and the initial attitude is determined, the \pyth{align} method will raise an error if called again.

\subsection{Compensating for deterministic errors and computation of Earth parameters}
The \pyth{compensate_errors_and_compute_params} method compensates for deterministic errors in the accelerometer and gyroscope measurements and computes the Earth parameters at the current latitude.  Deterministic errors in the accelerometer measurements are corrected by accounting for the bias, scale factor, and non-orthogonality of the accelerometer.  The corrected acceleration $\hat f$ is given by $$\hat f=(I+S_f+N_f)^{-1}(f-b_f)$$ where $f$ is the raw accelerometer measurement, $b_f$ is the accelerometer bias, $S_f$ is the scale factor matrix of the accelerometer, and $N$ is the non-orthogonality matrix of the accelerometer.  Similarly, deterministic errors in the gyroscope measurement are corrected via $$\hat \omega=(I+S_\omega+N_\omega)^{-1}(\omega-b_\omega).$$

The local gravity $g$ is calculated using the formula  $$g=a_1(1+a_2\sin^2\phi+a_3\sin^4\phi)+(a_4+a_5\sin^2\phi)h+a_6h^2$$ derived by Heiskanen and Moritz (1967), where $\phi$ is the current latitude, $h$ is the height above the ellipsoid, and the $a_i$ are coefficients given by GRS 80.

The radii of curvature of the Earth in the prime vertical $N$ and meridian $M$ directions then computed as 
$$N=\frac{a}{\sqrt{1-e^2\sin^2\phi}}$$
$$M=N\cdot \frac{1-e^2}{1-e^2\sin^2\phi}$$
where $e$ is the eccentricity of the Earth and $a$ is the semi-major axis.





\section{References}

\begin{small}

%\noindent \hangindent=0.5cm [1] A. Angrisano, ``GNSS/INS Integration Methods'', Ph.D. dissertation, Dept. Appl. Sci., University of Naples Parthenope, Naples, IT, 2010.

%\noindent \hangindent=0.5cm [2] Y. Yang, ``Tightly Coupled MEMS INS/GPS Integration with INS Aided Receiver Tracking Loops'', Ph.D. dissertation, Dept. Geo. Eng., University of Calgary, Calgary, AB, 2008.

\end{small}

\end{document}